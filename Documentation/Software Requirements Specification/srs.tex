\documentclass[11pt]{article}

\usepackage{color}
\usepackage{xcolor}
\usepackage{enumitem}
\usepackage{scrextend}
\usepackage{fancyhdr}
\usepackage{blindtext}
\usepackage{graphicx}
\usepackage{placeins}
\usepackage[utf8]{inputenc}
\usepackage[obeyspaces]{url}
\usepackage[margin=1.25in]{geometry}

\pagestyle{fancy}
\lfoot{\small\scshape SDP}
\cfoot{}
\rfoot{\footnotesize \thepage}
\lhead{\small\scshape CSCI 3428}
\chead{}
\rhead{\footnotesize Software Engineering}
\renewcommand{\headrulewidth}{.3pt}
\renewcommand{\footrulewidth}{.3pt}
\setlength\voffset{-0.25in}
\setlength\textheight{648pt}

\begin{document}

\title{CSCI 3428 - Software Requirements Specification}
\author{Group 4}
\date{Wednesday 23\textsuperscript{rd} October 2019}
\maketitle

\fancypagestyle{plain}{
\fancyhf{} % clear all header and footer fields
\fancyfoot[r]{\footnotesize \thepage} % except the right
\fancyfoot[l]{\small\scshape SRS} % and left
\renewcommand{\headrulewidth}{0pt}
\renewcommand{\footrulewidth}{.3pt}}

\section{Introduction}
\subsection{Purpose}
The program is intended to allow users to communicate with each other via text and images through
instant messaging. It distinguishes itself from other messaging platforms by prioritising
accessibility (by being tailored to the individual needs of each of the users), as well as
ease-of-use and simplicity. It hopes to respond to the need for simple and accessible web-based
services for use by the elderly.

\subsection{Intended Audience}
The program is being custom-designed for three residents of the Northwood Long-Term Care
facility in Halifax, Nova Scotia. While the program's functionality is similar in nature to any
other messaging platform, and can therefore be exploited by a wider user-group, its design will be
constrained according to the needs of the three residents, and will be driven based on the feedback
we receive from the residents during the testing and prototype phase.

\subsection{Intended Use}
The program is intended to be used as a text- and image- based communication platform. It will not
include functionality for voice or video communication between users, however it might implement
accessibility features that allow the users to interact with the program by voice, depending on
their specific needs.

\section{Description}
\textit{This section is still under development}
\subsection{User Needs}
\textit{This section is still under development}
\section{System Features and Requirements}
\subsection{External Interface Requirements}
\subsubsection{User Interfaces}
There are two primary user-interfaces that the users will interact with. The first is a log-in
screen, which allows us to distinguish between users, and gives each user access to their own
conversation list. Depending on the user's needs, it may not be required to use a password to
authenticate the log-in. The second is the conversation panel, which lists on the left all active
conversations that user has. The currently selected conversation appears on the right, and allows
the user to scroll through their entire conversation history, as well as toggle between viewing the
entire conversation, and only the images they have sent or received. This toggle is activated by
clicking the image icon that appears in the top-right corner of the chat window.

\begin{figure}[!htb]
  \includegraphics{login}
  \caption{Log-In Page}
\end{figure}
\begin{figure}
    \includegraphics{chat}
    \caption{Conversation Panel}
\end{figure}
\FloatBarrier

\subsubsection{Software Interfaces}
The product will be browser-based, and therefore can be used on any hardware with a browser that
supports the implementation languages. The product will communicate with a MySQL database that will
be hosted on the \url{ugdev.cs.smu.ca} server.

\subsubsection{Communication Interfaces}
To make use of the product, an active internet connection is required. The MySQL database will be
hosted on the \url{ugdev.cs.smu.ca} server.

\subsection{Functional Requirements}

\begin{figure}[!htb]
  \includegraphics{tree}
  \caption{Use-case diagram}
\end{figure}

\subsubsection{Essential}
\begin{enumerate}
    \item Authenticate and log-in user into system
    \item Allow user to choose and change who they communicate with
    \item Allow user to send text messages to others
    \item Allow user to send images to others
    \item Allow user to receive text messages from others
    \item ALlow user to receive images from others
    \item Display error message if connection to server fails
    \item Enable user to log-out of system
\end{enumerate}

\subsubsection{Desirable}
\begin{enumerate}
    \item Allow user to access a settings page
    \item Allow user to change font size for messaging interface
\end{enumerate}

\subsubsection{Optional}
\begin{enumerate}
    \item Allow user to change background colour of messaging interface
    \item Allow user to change text colour of messaging interface
\end{enumerate}

\subsection{Performance Requirements}
The messaging system will be browser-based, and run from the \url{ugdev.cs.smu.ca} server. Initial
load time will be dependent on the internet connection available to the user, and the stability of
the hosting server. The performance will additionally be marginally dependent on the hardware
available to the user to interact with the program.

\subsection{Design Constraints}
Users must have access to a modern web browser (e.g. Google Chrome, Mozilla Firefox, Safari, or
Internet Explorer) that is able to support the program. This includes being compatible with HTML5,
CSS3, and being able to run JavaScript- or Python-based scripts. Additionally, the user will need an
active internet connection to be able to make use of the program. From the development perspective,
the program is constrained according to compatibility with the \url{ugdev.cs.smu.ca} server. That
includes using databases that are available to be installed on the server.

\end{document}
